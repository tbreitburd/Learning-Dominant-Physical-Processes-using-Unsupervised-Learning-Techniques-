. Introduction (100 words)

	•	Briefly introduce the project and its significance.
	•	Mention the core aim: evaluating the reproducibility and robustness of the Callaham et al. (2021) method.

2. Objectives and Motivation (150 words)

	•	Explain the importance of reproducibility in scientific research.
	•	Highlight the problem of complex differential equations in engineering and physical sciences.
	•	Discuss the motivation to simplify these equations using dominant balance or scale analysis and the novel approach proposed by Callaham et al.

3. Methodology (250 words)

	•	Summarize the three-step method proposed by Callaham et al.:
	1.	Data & Equation Space Representation
	2.	Gaussian Mixture Model (GMM) Clustering
	3.	Sparse Principal Component Analysis (SPCA)
	•	Briefly describe how each step contributes to identifying dominant balance models.
	•	Mention any modifications or additional algorithms tested, such as Spectral Clustering, K-Means, and Weighted K-Means.

Figures to include:

	•	Figure 4.1: Example of a Gaussian Mixture Model fit to data.
	•	Figure 4.2: Example of a 2D dataset projected onto its first 2 principal components.

4. Conducted Research and Results (350 words)

	•	Discuss the portability and reproducibility of the code, including any issues encountered and resolved.
	•	Detail the turbulent boundary layer case:
	•	How the original and alternative codes were used to reproduce the results.
	•	Key findings from the reproduction of the RANS equation’s terms and GMM clustering.
	•	Results from applying SPCA to identify active terms.
	•	Summarize the exploration of other algorithms:
	•	Results from Spectral Clustering, K-Means, and Weighted K-Means.
	•	Highlight the stability assessment:
	•	Impact of different numbers of clusters and alpha values on the results.
	•	Effect of training set size on the identified balance models.

Figures to include:

	•	Figure 5.1: Plot of the 6 terms in the RANS equation using the original Callaham code.
	•	Figure 5.2: Plot of the 6 terms in the RANS equation using alternative code.
	•	Figure 5.3: Covariance matrices for each cluster found by the sklearn and custom GMM algorithms.
	•	Figure 5.4: Plot of the clusters found by the sklearn and custom GMM algorithms.
	•	Figure 5.5: Plot of unique balance models found after applying SPCA with the original and alternative code.
	•	Figure 5.6: Plot of unique balance model clusters in physical space with the original and alternative code.

5. Application to Elasto-Inertial Turbulence (100 words)

	•	Briefly describe the application of the method to a new dataset involving elasto-inertial turbulence.
	•	Highlight the potential of the method to uncover new dominant balance regimes in complex flows.

Figures to include:

	•	Figure 6.1: Plot of the attractors found in the DNS of the FENE-P model.

6. Conclusion (100 words)

	•	Summarize the key findings and their implications.
	•	Reflect on the strengths and limitations of the method.
	•	Suggest potential future work or improvements.
