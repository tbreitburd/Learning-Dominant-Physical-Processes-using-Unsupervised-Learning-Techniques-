\documentclass[12pt]{report} % Increased the font size to 12pt
\usepackage{epigraph}
\usepackage{geometry}

% Optional: customize the style of epigraphs
\setlength{\epigraphwidth}{0.5\textwidth} % Adjust the width of the epigraph
\renewcommand{\epigraphflush}{flushright} % Align the epigraph to the right
\renewcommand{\epigraphrule}{0pt} % No horizontal rule
\usepackage[most]{tcolorbox}
\usepackage{amsmath, amssymb, amsthm}
\usepackage{graphicx}
\usepackage[utf8]{inputenc}
\usepackage{hyperref} % Added for hyperlinks
\usepackage{listings} % Added for code listings
\usepackage{color}    % Added for color definitions
\usepackage[super]{nth}
\usepackage{fancyhdr}
\usepackage{tikz}
\usepackage{cite}
\usetikzlibrary{shapes.geometric, arrows, positioning}

\tikzstyle{startstop} = [rectangle, rounded corners, text centered, draw=black, fill=red!30]
\tikzstyle{io} = [trapezium, trapezium left angle=70, trapezium right angle=110, text centered, draw=black, fill=blue!30]
\tikzstyle{process} = [rectangle, text centered, draw=black, fill=orange!30]
\tikzstyle{decision} = [diamond, text centered, draw=black, fill=green!30]
\tikzstyle{arrow} = [thick,->,>=stealth]

% Define the header and footer for general pages
\pagestyle{fancy}
\fancyhf{} % Clear all header and footer fields
\fancyhead{} % Initially, the header is empty
\fancyfoot[C]{\thepage} % Page number at the center of the footer
\renewcommand{\headrulewidth}{0pt} % No header line on the first page of chapters
\renewcommand{\footrulewidth}{0pt} % No footer line

% Define the plain page style for chapter starting pages
\fancypagestyle{plain}{%
  \fancyhf{} % Clear all header and footer fields
  \fancyfoot[C]{\thepage} % Page number at the center of the footer
  \renewcommand{\headrulewidth}{0pt} % No header line
}

% Apply the 'fancy' style to subsequent pages in a chapter
\renewcommand{\chaptermark}[1]{%
  \markboth{\MakeUppercase{#1}}{}%
}

% Redefine the 'plain' style for the first page of chapters
\fancypagestyle{plain}{%
  \fancyhf{}%
  \fancyfoot[C]{\thepage}%
  \renewcommand{\headrulewidth}{0pt}%
}

% Header settings for normal pages (not the first page of a chapter)
\fancyhead[L]{\slshape \nouppercase{\leftmark}} % Chapter title in the header
\renewcommand{\headrulewidth}{0.4pt} % Header line width on normal pages

\setlength{\headheight}{14.49998pt}
\addtolength{\topmargin}{-2.49998pt}
% Define colors for code listings
\definecolor{codegreen}{rgb}{0,0.6,0}
\definecolor{codegray}{rgb}{0.5,0.5,0.5}
\definecolor{codepurple}{rgb}{0.58,0,0.82}
\definecolor{backcolour}{rgb}{0.95,0.95,0.92}

% Setup for code listings
\lstdefinestyle{mystyle}{
    backgroundcolor=\color{backcolour},
    commentstyle=\color{codegreen},
    keywordstyle=\color{magenta},
    numberstyle=\tiny\color{codegray},
    stringstyle=\color{codepurple},
    basicstyle=\ttfamily\footnotesize,
    breakatwhitespace=false,
    breaklines=true,
    captionpos=b,
    keepspaces=true,
    numbers=left,
    numbersep=5pt,
    showspaces=false,
    showstringspaces=false,
    showtabs=false,
    tabsize=2
}

\lstset{style=mystyle}

% Definition of the tcolorbox for definitions
\newtcolorbox{definitionbox}[1]{
  colback=red!5!white,
  colframe=red!75!black,
  colbacktitle=red!85!black,
  title=#1,
  fonttitle=\bfseries,
  enhanced,
}

% Definition of the tcolorbox for remarks
\newtcolorbox{remarkbox}[1]{
  colback=blue!5!white,     % Light blue background
  colframe=blue!75!black,   % Darker blue frame
  colbacktitle=blue!85!black, % Even darker blue for the title background
  title=#1,            % Title text for remark box
  fonttitle=\bfseries,      % Bold title font
  enhanced,
}

% Definition of the tcolorbox for examples
\newtcolorbox{examplebox}[1]{
  colback=green!5!white,    % Light green background
  colframe=green!75!black,   % Darker green frame
  colbacktitle=green!85!black,  % Even darker green for the title background
  title=#1,         % Title text for example box
  fonttitle=\bfseries,    % Bold title font
  enhanced,
}

% Definitions and examples will be put in these environments
\newenvironment{definition}
    {\begin{definitionbox}}
    {\end{definitionbox}}

\newenvironment{example}
    {\begin{examplebox}}
    {\end{examplebox}}

\geometry{top=1.5in} % Adjust the value as needed
% ----------------------------------------------------------------



\begin{document}

\begin{titlepage}
  \centering
  \vspace*{2cm}
  {\LARGE\bfseries MPhil DIS Report 24\par}
  \vspace{1cm}
  {\Large\itshape\ CRSiD:\ tmb76\par}
  \vspace{1cm}
  {\Large\itshape\ University of Cambridge\par}
  \vfill
  {\large\today\par}
\end{titlepage}

\tableofcontents

\chapter{Executive Summary}

In many areas of physics, the equations describing the observed are complex and include a large number of terms. For general use and numerical simulations, it is often necessary to simplify these equations. This report investigates the work carried out in the Callaham paper, which explores the use of unsupervised learning to identify dominant balance regimes from simulated data of the terms in the equation.

\chapter{Introduction}

For many problems in engineering of physical sciences, equations involve a large number of terms and are complex differential equations. Simulating them is therefore computationally expensive. However, it is often the case that these systems have multiple asymptotic local behaviours where the system is dominated only by a few terms. In these cases, one can simplify the equations to a balance between these dominant terms. This method is known as dominant balance or scale analysis, and has been a powerful tool in physics. Though it is extremely useful, it also requires a lot of expertise and has been done by hand in the past. In this report, a novel approach, developped by Callaham et al. (2021)\cite{callaham2021learning} is discussed, verified and used on new data. The first part of the report will be a discussion of the background of the paper and of the method used. Then, the research and results of the Callaham et al. (2021)\cite{callaham2021learning} paper will be shortly investigated, evaluating the ease of reproducibility of their work. Third, focussing on one of the examples used in the paper, use of other algorithms than the one used in the paper will be carried out, to test the robustness of the method. Finally, the method will be used on a new dataset, from simulation of elasto-inertial turbulence, a property of of polymer laden flow.


\chapter{Background}


Motivation for the work. Why is it important to be able to identify the dominant balance regimes in equations? What are the challenges but also what's the same in every dominant balance regardless of physics type?



Past litterature if any. Has there been models with similar objectives but less flexibility (like only for one of the examples)





\chapter{Methodology}


Get simulated data of the terms in the equation of physical variables from which terms in the equation can be derived

Group the data into feature space, with each term as a feature

Cluster the data using GMM

SPCA to identify which terms are active in each cluster

Group together clusters that have the same active terms


\chapter{Conducted research}

Callaham has used the algorithm in a few different cases to validate how it functions.



\section{Portability of the code}

\section{Reproducibility of the results}

\section{Exploration of other algorithms}

\subsection{Spectral clustering}

\subsection{K-Means}

\subsection{Weighted K-Means}

\section{Stability Assessment}

\subsection{Under different number of clusters set}

\subsection{Under different training set size}

\chapter{Elasto-inertial turbulence}

\section{Background}

\section{Methodology}

\section{Results}

\section{Discussion}

\chapter{Data analysis pipeline}


\bibliographystyle{plain}
\bibliography{refs.bib}

\end{document}
